\section{Analiza czynników wpływających}
\indent \indent Internet stanowi aktualnie popularne medium służące do wymiany 
oraz sprzedaży informacji, dóbr i usług. 
Przyciągnięcie użytkowników do komercyjnych stron internetowych, nie stanowi 
najważniejszego celu.
Równie ważne jest zatrzymanie odwiedzających oraz nakłonienie ich do zrealizowania 
transakcji za pośrednictwem wspomnianych stron, a więc dokonanie konwersji na 
klientów \cite{helander}\myfoot[helander].
Jednym z czynników, który jest kluczowy dla generowania ruchu oraz sprzedaży jest 
interfejs użytkownika. Temat interfejsu użytkownika był wielokrotnie poruszany 
w badaniach naukowych \cite{spiller, bellman, keeney, yoo, kim}
\myfoot[spiller]\myfoot[bellman]\myfoot[keeney]\myfoot[yoo]\myfoot[bellman].
Źle zorganizowane i mylące strony, nieintyicyjne ikony, nieprzemyślany system 
nawigacji na stronie i źle zaprojektowane menu mogą mieć niekorzystny wpływ na 
decyzje zakupowe potencjalnych klientów.
Nieświadomość podstawowych zasad rządzących usability mogą mieć znaczenie nie 
tylko w kontekście wygody i łatwości użytkownia, ale też może prowadzić do 
dezinformacji w temacie oferowanych produktów, ich cen, dostępności oraz 
sposobu poruszania po sklepie.
Co więcej są też takie aspekty wpływające na decyzje zakupowe w kontekście 
samego działania sklepów internetowych, jak wydajność i szybkość działania, 
obsługa klienta, szybkość realizacji zamówień oraz wiele innych mniej znaczących. 
W tym rozdziale zostaną opisane najważniejsze z tych czynników.

\subsection{Wydajność}

\subsection{Usability}

\subsection{Quality assurance}

\subsection{Cross-selling i up-selling}

\newpage
