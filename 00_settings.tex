% ------------------------------------------------------------------------
% PACZKI
% ------------------------------------------------------------------------
%kodowanie i obsługa języka
\usepackage{textcomp}
\usepackage[utf8]{inputenc}
\usepackage[T1]{fontenc}
\usepackage[english,polish]{babel}
\usepackage{polski}
%czcionki
\usepackage{times} %Times new Romanpa
%\usepackage[scaled]{uarial}
%obsługa pdfa
\usepackage[pdftex,usenames,dvipsnames]{color}      %Obsługa kolorów. Opcje usenames i dvipsnames wprowadzają 
%bibliografia
\usepackage{hyperref}
\usepackage[numbers,sort&compress]{natbib} %Porządkuje zawartość odnośników do literatury, np. [2-4,6]. Musi być pod pdf'em, a styl bibliogfafii musi mieć nazwę z dodatkiem 'nat', np. \bibliographystyle{unsrtnat} (w kolejności cytowania).
\usepackage{hypernat} %Potrzebna pakietowi natbib do wspolpracy z pakietem hyperref (wazna kolejnosc: 1. hyperref, 2. natbib, 3. hypernat).
%grafika i geometria strony
\usepackage{extsizes} %Dostepne inne rozmiary czcionek, np. 14 w poleceniu: \documentclass[14pt]{article}.
\usepackage[final]{graphicx}
\graphicspath{{img/}}
%inne
\usepackage{expdlist}    %Zmienia definicję środowiska description, daje większe możliwości wpływu na wygląd listy.
\usepackage{flafter}     %Wprowadza parametr [tb] do polecenia \suppressfloats[t] (polecenie to powoduje nie umieszczanie rysunkow, tabel itp. na stronach, na ktorych jest to polecenie (np. moze byc to stroma z tytulem rozdzialu, ktory chcemy zeby byl u samej gory, a nie np. pod rysunkiem)).
\usepackage{array}       %Ładniej drukuje tabelki (np. daje wiecej miejsca w komorkach -- nie są tak sciemnione, jak bez tego pakietu).
\usepackage{listings}    %Listingi programow.
\usepackage[format=hang,labelsep=period,labelfont={bf,small},textfont=small]{caption}   %Formatuje podpisy pod rysunkami i tabelami. Parametr 'hang' powoduje wcięcie kolejnych linii podpisu na szerokosc nazwy podpisu, np. 'Rysunek 1.'.
\usepackage{appendix} %załączniki.
\usepackage{fancyhdr} %drukowanie dwustronne z lustrzaną numeracją
\usepackage[a4paper,left=3cm,right=1.5cm,top=2.5cm,bottom=2.5cm]{geometry}
\usepackage{sectsty} %modyfikacje w sekcjach
% ------------------------------------------------------------------------
% USTAWIENIA
% ------------------------------------------------------------------------
% numeracja w spisie treści
\makeatletter
\renewcommand\l@section[2]{%
  \ifnum \c@tocdepth >\z@
    \addpenalty\@secpenalty
    \addvspace{1.0em \@plus\p@}%
    \setlength\@tempdima{1.5em}%
    \begingroup
      \parindent \z@ \rightskip \@pnumwidth
      \parfillskip -\@pnumwidth
      \leavevmode \bfseries
      \advance\leftskip\@tempdima
      \hskip -\leftskip
      #1\nobreak\ 
      \leaders\hbox{$\m@th\mkern \@dotsep mu\hbox{.}\mkern \@dotsep mu$}
     \hfil \nobreak\hb@xt@\@pnumwidth{\hss #2}\par
    \endgroup
  \fi}
\makeatother



\setlength{\parskip}{6mm}
\frenchspacing %normalne odstępy w tekscie 
\setlength{\parskip}{3pt} %odstęp między akapitami
\setlength{\parindent}{28pt}
\linespread{1.15} %interlinia
\bibliographystyle{plainnat}

\pagestyle{fancy}
\fancyhf{}
\renewcommand{\headrulewidth}{0pt}
\fancyfoot[LE,RO]{\thepage}

\renewcommand{\rmdefault}{phv} % Arial
\renewcommand{\sfdefault}{phv} % Arial

%%sekcje
\sectionfont{\Large\MakeUppercase}
\subsectionfont{\fontsize{13}{18pt}}
\subsectionfont{\normalsize}

%%cytowanie
\def \myfoot[#1]{\let\thefootnote\relax\footnotetext{\citetalias{#1}}}

%cytowanie literatury
\defcitealias{alreck_settle}{\cite{alreck_settle} Alreck et al.: Gender Effects on Internet, Catalogue and Store Shopping, 2002}
\defcitealias{stafford}{\cite{stafford} Stafford et al.: International and Cross-Cultural Influences on Online Shopping Behavior, 2004}
\defcitealias{susskind}{\cite{susskind} Susskind: Electronic Commerce and World Wide Web Apprehensiveness, 2004}
\defcitealias{li}{\cite{li} Li et al.: The Impact of Perceived Channel Utilities, Shopping Orientations, and Demographics on the Consumer's Online Buying Behavior, 1999}
\defcitealias{alsomali}{\cite{alsomali} Al-Somali et al.: An investigation into the acceptance of online banking in Saudi Arabia, 2009}
\defcitealias{lu}{\cite{lu} Lu et al.: Technology acceptance model for wireless internet, 2003}
\defcitealias{mahmood}{\cite{mahmood} Mahmood et al.: On-Line Shopping Behavior, 2004}
\defcitealias{wee_ramacha}{\cite{wee_ramacha} Wee et al.: Cyberbuying in China, Hong Kong and Singapore, 2000}
\defcitealias{park_2004}{\cite{park_2004} Park et al.: Risk-Focused E-Commerce Adoption Model, 2004}
\defcitealias{joines}{\cite{joines} Joines et al.: Exploring Motivations for Consumer Web Use and Their Implications for E-Commerce, 2003}
\defcitealias{garbarino}{\cite{garbarino} Garbarino et al.: Gender Differences in the Perceived Risk of Buying Online and the Effects of Receiving a Site Recommendation, 2004}
\defcitealias{lynch}{\cite{lynch} Lynch et al.: The Global Internet Shopper, 2001}
\defcitealias{xia}{\cite{xia} Xia: Affect As Information, 2002}
\defcitealias{shin}{\cite{shin} Shin: Towards an understanding of the consumer acceptance of mobile wallet, 2009}
\defcitealias{eurostat}{\cite{eurostat} Eurostat: Internet usage in 2009 – households and individuals, 2009}
\defcitealias{slyke}{\cite{slyke} Slyke: Gender Differences in Perceptions of Web-Based Shopping, 2002}
\defcitealias{kantor}{\cite{kantor} Kantor: Electronic Data Interchange (EDI), 1996}
\defcitealias{gate}{\cite{gate} Gate: Electronic Funds Transfer, 2004}
\defcitealias{swatman}{\cite{swatman} Swatman: Electronic Commerce: Origins and Future Directions, 1996}
\defcitealias{kaushik}{\cite{kaushik} Kaushik: History of E-Commerce, 2010}
\defcitealias{census}{\cite{census} QUARTERLY RETAIL E-COMMERCE SALES 4th QUARTER 2008}
\defcitealias{helander}{\cite{helander} Helander: Modeling the customer in electronic commerce, 2000}
\defcitealias{spiller}{\cite{spiller} Spiller: A classification of internet retail stores, 1997}
\defcitealias{bellman}{\cite{bellman} Bellman: Predictors of online buying behavior, 1999}
\defcitealias{keeney}{\cite{keeney} Keeney: The value of Interncet commerce to the customer, 1999}
\defcitealias{yoo}{\cite{yoo} Yoo: Experiment on the effectiveness of link structure for convenient cybershopping, 2000}
\defcitealias{kim}{\cite{kim} Kim: Critical design factors for successful e-commerce systems, 2002}
\defcitealias{hoffman}{\cite{hoffman} Hoffman: Commercial Scernarios for the Web: Opportunities and challenges, 1996}
\defcitealias{cappel}{\cite{cappel} Cappel: World Wide Web uses for electronic commerce: Toward a classification scheme, 1996}
\defcitealias{gregor}{\cite{gregor} Gregor: e-Commerce, 2002}
\defcitealias{global_trends}{\cite{global_trends} Global Trends in Online Shopping, 2010}
\defcitealias{constantinades}{\cite{constantinades} Influencing the online consumer's behavior: the Web experience, 2004}
\defcitealias{nielsen} {\cite{nielsen} Nielsen et al.: E-Commerce user experience, 2001}
\defcitealias{pitkow_kehoe} {\cite{pitkow_kehoe} Pitkow: Emerging trends in the www user population, 1996}
\defcitealias{katz_aspen} {\cite{katz_aspen} Katz: Motives, hurdles and dropouts. Who is on the Internet and why?, 1997}
\defcitealias{manhartsberger} {\cite{manhartsberger} Manhartsberger: Web Usability: Das Prinzip des Vertrauens, 2001}
\defcitealias{gamification} {\cite{gamification} Wikipedia: Gamification}
\defcitealias{zichermann} {\cite{zichermann} Zichermann: Game Based Marketing, 2010}
\defcitealias{harris} {\cite{harris} Harris: Privacy leadership initiative, 2001}
\defcitealias{grabner} {\cite{grabner} Grabner-Kräuter: Empirical research in online trust: a review and critical assessment, 2003}

\def\literatura{
  \section{LITERATURA}
  \renewcommand*{\refname}{}
  \subsection{Bibliografia}
  \vspace{-40pt}
    \begin{thebibliography}{5}
    \bibitem{alsomali} Al-Somali, S., Gholami, R. i Clegg, B.: An investigation into the acceptance of online banking in Saudi Arabia, Technovation, wydanie 29, nr 2, 2009
    \bibitem{alreck_settle} Alreck, P. and Settle, R. B.: Gender Effects on Internet, Catalogue and Store Shopping, Journal of Database Marketing, wydanie 9, nr 2, strony 150-162, 2002
    \bibitem{bellman} Bellman S., Lohse G. L., Johnson E. J.: Predictors of online buying behavior, Communications of ACM, wydanie 42, str. 32-38, 1999
    \bibitem{cappel} Cappel J. J. i Myerscough M. A.: World Wide Web uses for electronic commerce: Toward a classification scheme, 1996
    \bibitem{constantinades} Constantinades E.: Influencing the online consumer's behavior: the Web experience, Internet Research, wydanie 14, nr 2, str. 111-126, 2004
    \bibitem{eurostat} Eurostat, Internet usage in 2009 – households and individuals, 2009
    \bibitem{gamification} Wikipedia, Gamification, http://en.wikipedia.org/wiki/Gamification, dostęp 06.01.2012
    \bibitem{garbarino} Garbarino, E. and Strabilevitz, M.: Gender Differences in the Perceived Risk of Buying Online and the Effects of Receiving a Site Recommendation, Journal of Business Research wydanie 57, str. 768-775, 2004
    \bibitem{gate} Gate R., Nacamuli A.: Electronic Funds Transfer, The Internet Encyclopedia, 2004 
    \bibitem{global_trends} Global Trends in Online Shopping, A Nielsen Global Consumer Report, June 2010
    \bibitem{grabner} Grabner-Kräuter S., Kaluscha A.E.: Empirical research in online trust: a review and critical assessment, International Journal of Human-Computer Studies, wydanie 58, nr 6, str. 783-812, 2003
    \bibitem{gregor} Gregor B., Stawiszyński M.: e-Commerce, 2002
    \bibitem{harris} Harris Interactive: Privacy leadership initiative, 2001
    \bibitem{helander} Helander M. G., Khalid H. M.: Modeling the customer in electronic commerce. Applied Ergonomics wydanie 31, str. 609-619, 2000
    \bibitem{hoffman} Hoffman D. L., Novak T. P. i Chatterjee P.: Commercial Scernarios for the Web: Opportunities and challenges. Journal of Computer-Mediated Communication, Special Issue on Electronic Commerce, str. 1-3, 1996
    \bibitem{joines} Joines, J., Scherer, C. and Scheufele, D.: Exploring Motivations for Consumer Web Use and Their Implications for E-Commerce, Journal of Consumer Marketing wydanie 20, nr 2, str. 90-109, 2003
    \bibitem{kantor} Kantor M., Burrows J. H.: Electronic Data Interchange (EDI), National Institute of Standards and Technology, 1996
    \bibitem{katz_aspen} Katz J., Aspen P.: Motives, hurdles and dropouts. Who is on the Internet and why? Communications of the ACM, wydanie 40, str. 97-102, 1997
    \bibitem{kaushik} Kaushik P.: History of E-Commerce, 2010
    \bibitem{keeney} Keeney R. L.: The value of Internet commerce to the customer, Management Science, wydanie 45, str. 533-542, 1999
    \bibitem{kim} Kim J., Lee J.:Critical design factors for successful e-commerce systems, Behaviour and Information Technology, wydanie 21, str. 185-199, 2002
    \bibitem{li} Li, H., Kuo, C. and Russell, M. G.: The Impact of Perceived Channel Utilities, Shopping Orientations, and Demographics on the Consumer's Online Buying Behavior, Journal of Computer-Mediated Communication, wydanie 5, nr 2, 1999
    \bibitem{lu} Lu, J., Yu, C.S., Liu, C. and Yao, J.: Technology acceptance model for wireless internet, Internet Research: Electronic Networking Applications and Policy, wydanie 13 nr 3, str. 206-22, 2003
    \bibitem{lynch} Lynch, P. D., Kent, R. J. and Srinivasan, S. S.: The Global Internet Shopper: Evidence From Shopping Tasks in Twelve Countries, Journal of Advertising Research wydanie 41, nr 3, str. 15-23, 2001
    \bibitem{mahmood} Mahmood, M. A., Bagchi, K. and Ford, T. C.: On-Line Shopping Behavior: Cross-Country Empirical Research, International Journal of Electronic Commerce, rocznik 9, nr 1, str. 9-30, 2004
    \bibitem{manhartsberger} Manhartsberger M., Musil S.: Web Usability: Das Prinzip des Vertrauens, Galileo Press, Bonn, 2001
    \bibitem{nielsen} Nielsen J., Molich R., Snyder C., Farrel S.: E-Commerce user experience, Fremont, CA: Nielsen Norman Group, 2001
    \bibitem{park_2004} Park, J., Lee, D. and Ahn, J.: Risk-Focused E-Commerce Adoption Model: a Cross-Country Study, Journal of Global Information Management wydanie 7, str. 6-30, 2004
    \bibitem{census} QUARTERLY RETAIL E-COMMERCE SALES 4th QUARTER 2008, http://www.census.gov/mrts/www/data/html/08Q4.html, dostęp 17.09.2011
    \bibitem{pitkow_kehoe} Pitkow J.E., Kehoe C.M.: Emerging trends in the www user population . Communications of the ACM, wydanie 39, str. 106-108, C.M. 1996
    \bibitem{shin} Shin, D.H.: Towards an understanding of the consumer acceptance of mobile wallet,
Computers in Human Behavior, wydanie 25, nr 6, str. 1343-54, 2009
    \bibitem{slyke} Slyke, C. V., Comunale, C. L. and Belanger, F.: Gender Differences in Perceptions of Web-Based Shopping, Communications of the ACM wydanie 45, nr 7, str. 82-86, 2002
    \bibitem{spiller} Spiller P., Lohse G. L.: A classification of internet retail stores. International Journal of Electronic Commerce, wydanie 2, str. 29-56, 1997
    \bibitem{stafford} Stafford, T. F., Turan, A. and Raisinghani, M. S.: International and Cross-Cultural Influences on Online Shopping Behavior, Journal of Global Information Management, wydanie 7, nr 2, str. 70-87, 2004
    \bibitem{susskind} Susskind, A.:Electronic Commerce and World Wide Web Apprehensiveness: An Examination of Consumers' Perceptions of the World Wide Web, Journal of Computer-Mediated Communication, rocznik 9, nr 3, 2004
    \bibitem{swatman} Swatman, P.M.C.: Electronic Commerce: Origins and Future Directions, 1st
Australian DAMA Conference, Melbourne, Victoria, 1996
    \bibitem{wee_ramacha} Wee, K. N. L. and Ramachandra, R.: Cyberbuying in China, Hong Kong and Singapore: Tracking the Who, Where, Why and What Online Buying, International Journal of Retail \& Distribution Management wydanie 28, nr 7, str. 307-316, 2000
    \bibitem{xia} Xia, L.: Affect As Information: the Role of Affect in Consumer Online Behaviors, Advances in Consumer Research wydanie 29, nr 1, str. 93-100, 2002
    \bibitem{yoo} Yoo B., Kim J.: Experiment on the effectiveness of link structure for convenient cybershopping, Journal of Organizational Computing and Electronic Commerce, wydanie 10, str. 241-256, 2000
    \bibitem{zichermann} Zichermann G., Linder J.: Game-based marketing: inspire customer loyalty through re-wards, challenges, and contests, Wiley, 2010

  \end{thebibliography}
  %\renewcommand*{\refname}{}
  %\subsection{Źródła internetowe}
  %\vspace{-40pt}
  %\begin{thebibliography}{5}
  %  \bibitem{census} QUARTERLY RETAIL E-COMMERCE SALES 4th QUARTER 2008, http://www.census.gov/mrts/www/data/html/08Q4.html, dostęp 17.09.2011
  %\end{thebibliography}
  \newpage
}

