\section{ZAŁOŻENIA PROJEKTOWE}

\subsection{Założenie funkcjonalne}
W tym podrozdziale będzie zawarte cytowanie i odnośnik do literatury na końcu dokumentu. \\
{\emph{Il faut exiger de chacun ce que c
hacun peut donner, reprit le roi. L'autorité repose d'abord sur la raison. 
Si tu ordonnes à ton peuple d'aller se jeter à la mer, il fera la révolution. 
J'ai le droit d'exiger l'obéissance parce que mes ordres sont raisonnables.
Alors mon coucher de soleil ? rappela le petit prince qui jamais n'oubliait une 
question une fois qu'il l'avait posée. Ton coucher de soleil, tu l'auras. 
Je l'exigerai. Mais j'attendrai, dans ma science du gouvernement, que les 
conditions soient favorables.}

\suppressfloats[t]
To jest element treści. {\bfseries Ten element jest wyróżniony.} 
{\emph{I ten też jest wyróżniony, ale inaczej.}}
W tekście używam też skrótów takich jak CAD (ang. Computer Aided Design) - 
komputerowo wspomagane projektowanie. 
W następnych zdaniach nie muszę już rozwijać skrótu i mogę napisać CAD. \\

%\begin{figure}[h]
%    \centering
%    \includegraphics[width=9.82cm]{rysunek.pdf}\\
%    \caption {Testowy obrazek}
%\end{figure}

\subsection{Założenia jakościowe}

\subsection{Założenia organizacyjne i czasowo-przestrzenne}

\subsection{Założenia dotyczące bezpieczeństwa projektowanego systemu}

\subsection{Założenia eksploatacyjne}

\subsection{Założenia ekonomiczne}

\newpage
